\section{Introduction}


In the Bipolar Junction Transistor tutorials, we saw that the output Collector current of
the transistor is proportional to input current flowing into the Base terminal of the de-
vice, thereby making the bipolar transistor a \textbf{CURRENT} operated device (Beta model) as
a smaller current can be used to switch a larger load current.

The Field Effect Transistor, or simply FET however, uses the voltage that is applied to their
input terminal, called the Gate to control the current flowing through them resulting in
the output current being proportional to the input voltage. As their operation relies on an
electric field (hence the name field effect) generated by the input Gate voltage, this then
makes the Field Effect Transistor a \textbf{VOLTAGE} operated device.

The Field Effect Transistor has one major advantage over its standard bipolar transistor
cousins, in that their input impedance, (Rin) is very high, (thousands of Ohms), while the
BJT is comparatively low. This very high input impedance makes them very sensitive to
input voltage signals, but the price of this high sensitivity also means that they can be
easily damaged by static electricity.