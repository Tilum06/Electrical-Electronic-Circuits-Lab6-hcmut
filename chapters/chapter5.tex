\section{Altium Designer}

\subsection{Button Input}

Push buttons or switches connect two points in a circuit when you press them. In a system, button is a traditional input method. The figure bellow is an example of a simple button, which is the target of this exercise:

\begin{figure}[H]
    \centering
    \includegraphics[width=0.3\textwidth]{graphics/manual/ex51_example.png}
    \caption{Example of a button input}
    \label{fig:ex51_example}
\end{figure}

The schematic of this circuit is proposed as follows:

\begin{figure}[H]
    \centering
    \includegraphics[width=0.5\textwidth]{graphics/manual/ex51_sch.png}
    \caption{Button schematic in Altium}
    \label{fig:ex51_sch}
\end{figure}

\textbf{Students are proposed to implement the circuit in Altium Designer. The manual can be found in the same playlist with other manual videos. Please capture the screen to present the schematic as well as the layout of your PCB.}

\textbf{Simulation results:}

\begin{figure}[H]
    \centering
    \includegraphics[width=0.7\textwidth]{graphics/ex51_2d_top.png}
    \caption{PCB 2D layout- TOP layer}
    \label{fig:ex51_2d_top}
\end{figure}

\begin{figure}[H]
    \centering
    \includegraphics[width=0.7\textwidth]{graphics/ex51_2d_bot.png}
    \caption{PCB 2D layout- BOTTOM layer}
    \label{fig:ex51_2d_bot}
\end{figure}

\begin{figure}
    \centering
    \includegraphics[width=0.7\textwidth]{graphics/ex51_3d_top.png}
    \caption{PCB 3D layout- TOP layer}
    \label{fig:ex51_3d}
\end{figure}

\begin{figure}[H]
    \centering
    \includegraphics[width=0.7\textwidth]{graphics/ex51_3d_bot.png}
    \caption{PCB 3D layout- BOTTOM layer}
    \label{fig:ex51_3d_bot}
\end{figure}


\subsection{ADC Input}

The second type of the input signal is the ADC input value, which is a kind of sensing device. In this sensorpart, we use two opamps which are packed in one ICLM358. ICLM358 includes two opamps. A photoresistor (also known as a light-dependent resistor, LDR, or photo-conductive cell) is use to measure the light intensitive. It is a passive component that decreases resistance with respect to receiving luminosity (light) on the component’s sensitive surface. An example of this module can be found in the figure bellow:

\begin{figure}[H]
    \centering
    \includegraphics[width=0.3\textwidth]{graphics/manual/ex52_example.png}
    \caption{ An example of a light sensor}
    \label{fig:ex52_example}
\end{figure}

The schematic of this circuit is proposed as follows, which is based on a voltage follower
circuit:

\begin{figure}[H]
    \centering
    \includegraphics[width=0.5\textwidth]{graphics/manual/ex52_sch.png}
    \caption{Light sensor schematic in Altium}
    \label{fig:ex52_sch}
\end{figure}

\textbf{Students are proposed to implement the circuit in Altium Designer. The manual can be found in the same playlist with other manual videos. Please capture the screen to present the schematic as well as the layout of your PCB.}

\textbf{Simulation results:}

\begin{figure}[H]
    \centering
    \includegraphics[width=0.7\textwidth]{graphics/ex52_2d_top.png}
    \caption{PCB 2D layout- TOP layer}
    \label{fig:ex52_2d_top}
\end{figure}

\begin{figure}[H]
    \centering
    \includegraphics[width=0.7\textwidth]{graphics/ex52_2d_bot.png}
    \caption{PCB 2D layout- BOTTOM layer}
    \label{fig:ex52_2d_bot}
\end{figure}

\begin{figure}
    \centering
    \includegraphics[width=0.7\textwidth]{graphics/ex52_3d_top.png}
    \caption{PCB 3D layout- TOP layer}
    \label{fig:ex52_3d_top}
\end{figure}

\begin{figure}[H]
    \centering
    \includegraphics[width=0.7\textwidth]{graphics/ex52_3d_bot.png}
    \caption{PCB 3D layout- BOTTOM layer}
    \label{fig:ex52_3d_bot}
\end{figure}