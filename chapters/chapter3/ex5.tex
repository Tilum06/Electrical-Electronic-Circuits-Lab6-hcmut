\subsection{Low Pass Filter}

Low pass filter is a filter which passes all frequencies from 0Hz (DC current) to upper cut-off frequency $f_H$ and rejects any signals above this frequency. A picture to demonstrate a low pass filter behavior is shown in the figure bellow:

\begin{figure}[H]
    \centering
    \includegraphics[width=0.5\textwidth]{graphics/question/c3_ex5_principles.png}
    \caption{Low pass filter principles}
    \label{fig:chap3_question_ex5_principles}
\end{figure}

\noindent Similar to the closed loop configuration, there also 2 types of low pass filter, including the inverting and non-inverting low pass filter. The figure bellow is an inverting low pass filter. The cut-off frequency is determined by this equation:

\[
f_H = \frac{1}{2 \pi R_2 C}
\]

By applying the value of R2 = 10K Ohm and C = 1nF , the cut-off frequency is around 16K Hz. In order to see the results, students are proposed to run the AC Sweep simulation profile (\textbf{Linear Type, Start and Stop frequency are 1Hz and 50kHz, 200 points}), as follows:

\begin{figure}[H]
    \centering
    \includegraphics[width=0.6\textwidth]{graphics/question/c3_ex5_ilpf_sch.png}
    \caption{Inverting low pass filter}
    \label{fig:chap3_question_ex5_ilpf_sch}
\end{figure}

\vspace{1cm}

\begin{figure}[H]
    \centering
    \includegraphics[width=0.7\textwidth]{graphics/question/c3_e5_simprofile.png}
    \caption{AC Sweep simulation profile}
    \label{fig:chap3_question_ex5_simprofile}
\end{figure}

\vspace{1cm}

The final results can be archived like the figure bellow:

\begin{figure}[H]
    \centering
    \includegraphics[width=0.7\textwidth]{graphics/question/c3_e5_simresult.png}
    \caption{Simulation results}
    \label{fig:chap3_question_ex5_simresult}
\end{figure}

\noindent\textbf{It is said that the cut-off frequency point having the gain reduced 3dB.} The gain at 0Hz
is 10 (input voltage is 2V and output voltage is 20V), or $20\log(10)=20\ \mathrm{dB}$, meanwhile, the
gain at 16kHz is 7 (input voltage is 2V and output voltage is 14V), or $20\log(7)\approx16.9\ \mathrm{dB}$.

\noindent The second type of a low pass filter, the non-inverting configuration, is presented as follows:

\begin{figure}[H]
    \centering
    \includegraphics[width=0.6\textwidth]{graphics/question/c3_ex5_nilpf_sch.png}
    \caption{Non-inverting low pass filter}
    \label{fig:chap3_question_ex5_nlpf_sch}
\end{figure}

\textbf{Students are proposed to calculate the value of R and C to have the amplifier factor equal to 10 and the cut-off frequency is the same as the previous example. The simulation result with AC Sweep mode is required to plot in this report as well.}

\textbf{Calculation:}

To have the amplifier factor equal to 10, the ratio between R2 and R1 is calculated as follows:
\[A_v = 1 + \frac{R_2}{R_3} = 10 \Rightarrow \frac{R_2}{R_3} = 9\]
By choosing $R_2 = 9k\Omega$ and $R_3 = 1k\Omega$, the cut-off frequency is calculated as follows:
\[f_H = \frac{1}{2 \pi R_2 C} = 16kHz\] 
\[ \Longrightarrow  C = \frac{1}{2 \pi R_2 f_H} = \frac{1}{2 \pi \cdot 9k\Omega \cdot 16kHz} \approx 1.1nF\]

The simulation result is presented as follows:
\begin{figure}[H]
    \centering
    \includegraphics[width=0.8\textwidth]{graphics/ex5_sim.png}
    \caption{Simulation results of non-inverting low pass filter}
    \label{fig:chap3_ex5_sim}
\end{figure}

The gain at 0Hz is 10 (input voltage is 2V and output voltage is 20V), or $20\log(10)=20\ \mathrm{dB}$, meanwhile, the gain at 16kHz is 7 (input voltage is 2V and output voltage is 14V), or $20\log(7)\approx16.9\ \mathrm{dB}$.