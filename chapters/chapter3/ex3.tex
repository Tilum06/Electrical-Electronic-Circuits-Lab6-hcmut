\subsection{Voltage Follower with Gain}

This basic circuit is not limited to the unity-gain configuration. As with a non-buffered op-amp, you can insert resistors into the feedback path to create overall gain from the input to the load voltage. Here is the non-unity-gain version of the circuit:

\noindent Students are proposed to implement this circuit on PSPICE with input is 2V and the gain is 3. The voltage supply for the load side is 12VDC. Value of $R_{LOAD}$ is 1K.

\begin{figure}[H]
    \centering
    \includegraphics[width=0.6\textwidth]{graphics/question/c3_ex3.png}
    \caption{Opamp follower with gain for the output}
    \label{fig:chap3_question_ex3}
\end{figure}

The simulation results in PSPICE (bias configuration) are presented here. Moreover, a short explanations are required in this report to explain the gain of the output follower voltage.

\textbf{Simulation results:}

\begin{figure}[H]
    \centering
    \includegraphics[width=0.7\textwidth]{graphics/ex3_sim_1k2k.png}
    \caption{Simulation results with $R_1 = 1K\Omega$ and $R_2 = 2K\Omega$}
    \label{fig:chap3_ex3_sim_1k2k}
\end{figure} 

\begin{figure}[H]
    \centering
    \includegraphics[width=0.7\textwidth]{graphics/ex3_sim_5k10k.png}
    \caption{Simulation results with $R_1 = 5K\Omega$ and $R_2 = 10K\Omega$}
    \label{fig:chap3_ex3_sim_5k10k}
\end{figure}

\textbf{Explanation:}

Both simulations use the same input voltage $V_{in} = 2\,V$ and the same resistor ratio 
$\dfrac{R_2}{R_1} = 2$. Therefore, the voltage gain is:
\[
    A_v = 1 + \dfrac{R_2}{R_1} = 1 + 2 = 3
\]
So the output voltage is:
\[
    V_{out} = A_v \cdot V_{in} = 3 \cdot 2\,V = 6\,V   
\]
This result is confirmed by both simulations in \cref{fig:chap3_ex3_sim_1k2k} and \cref{fig:chap3_ex3_sim_5k10k}.